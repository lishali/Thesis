% (This is included by thesis.tex; you do not latex it by itself.)

\begin{abstract}

% The text of the abstract goes here.  If you need to use a \section
% command you will need to use \section*, \subsection*, etc. so that
% you don't get any numbering.  You probably won't be using any of
% these commands in the abstract anyway.

Graphs are a rich and fundamental object of study, of interest from both theoretical and applied points of view. This thesis is in two parts and gives a treatment of graphs from two differing points of view, with the goal of doing inference on graphs.  The first is a mathematical approach.  We create a formal framework to investigate the quality of inference on graphs given partial observations.  The proofs we give apply to all graphs without assumptions.  In the second part of this thesis, we take on the problem of clustering with the aid of deep neural networks and apply it to the problem of community detection. The results are competitive with the state of the art, even at the information theoretic threshold of recovery of community labels in the stochastic blockmodel. 
\end{abstract}
